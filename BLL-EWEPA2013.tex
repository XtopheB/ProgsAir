
\documentclass[slidestop,compress,mathserif, handout]{beamer}
%\usepackage[bars]{beamerthemetree} % Beamer theme v 2.2
%\usetheme{Antibes} % Beamer theme v 3.0
%\usetheme{Singapore}
\usetheme{Dresden}
\usecolortheme{seahorse}
%%%% Langue
%\usepackage[frenchb]{babel}
\usepackage[english]{babel}
%\usecolortheme[red] % Beamer color theme

\setbeamertemplate{navigation symbols}{\insertframenumber /
\inserttotalframenumber}
\definecolor{vert}{rgb}{0.1,0.7,0.2}
\definecolor{brique}{rgb}{0.7,0.16,0.16}
\definecolor{gris}{rgb}{0.2,0.4,0.3}



\title{Productivity and Efficiency of World Airlines:\\
An Empirical Application with order-m and $\alpha$-Frontiers}
%\title{Technical \& Efficiency Change in the French Food Industries}
\author[Bontemps \textit{et al.} (2012)]{Christophe Bontemps$^1$ , Steve Lawford$^2$, and Nathalie Lenoir$^2$}
\institute{(1) Toulouse School of Economics and (2) French Civil Aviation University, Toulouse}

\logo{\includegraphics[height=0.5cm]{Graphics/LogoTSE.jpg}}  % <------------------ A remettre
\date{\textbf{EWEPA}, Helsinki, \textit{ June 2013}}



\begin{document}
\frame{\titlepage}

%\frame{{\color{blue} \large ROADMAP} \tableofcontents}

%\section{Introduction}


\section{Airlines industry and data}
 \frame[t] {
 Very preliminary investigation on this industry
\begin{itemize}
%The processed food and beverage industry is Australia�s largest manufacturing industry with a turnover of more than $71.4 billion in 2005�06.
  \item Airline industry (world level)
  \item Original dataset collection (still in progress!) the "\textit{ENAC Air Transport Data}" collecting information on world airlines since  1995 (from IATA, ICAO and other sources).
  \item Previous study used country level data:
  \begin{itemize}
      \item UK (Assaf 2011);
      \item USA (Kumbhakar 1992, Oum \textit{et al.} 2005, F\"{a}re, Grosskopf \& Sickles 2007) or
      \item UK \textit{vs} USA (Assaf \& Josiassen 2012),
  \end{itemize}
  \item   while
  \begin{itemize}
       \item the market is global for (some?) inputs/output,
       \item competion is gloabl and
       \item no cross country study has been done yet.
    \end{itemize}
\end{itemize}
}


\subsection{Questions}
\frame[t] {
\begin{itemize}[<+->]
  \item Questions addressed in this (preliminary) paper:
  \bigskip
  \begin{itemize}
    \item What are the features of this industry efficiency at a world level ?
    \item[-] Which are the most efficient airlines ?
    \item[-] Are majors more efficient than low cost carriers (LCC) ?
    \item Do we observe ``changes'' over time?
    \item[-] Technical change over time?
    \item[-] Is the ranking of airlines evolving over time ?

  \end{itemize}
  \bigskip
  \item {Questions \color{brique}{not yet} addressed:}
  \bigskip
  \begin{itemize}
    \item Did the fuel efficiency of airline improve over time ?
    \item Did labour efficiency improve over time ?
    \item Is competition affecting productivity  ? i.e. Are the airlines who operate on routes with more competitors more efficient
    \item Do airlines involved in an alliance have a higher productivity ?
  \end{itemize}
\end{itemize}
}


\subsection{Data}
 \frame[t] {
  \vspace{1cm}
  Data (so far)
\begin{itemize}
  \item  We have 1 output $Y$ and 4 inputs ($K$, $L$, $E$, $M$)
  \begin{itemize}
      \item $Y$ is the airlines revenue (including passenger \& freight revenue, and also incidental revenue)
      \item $K$ is the Flying capital defined as the total fleet max take-off weight (in kg).
      \item $L$ is the total personnel (permanent + non-permanent) in pers.x month
      \item $E$ is the total fuel quantity (in Gallons)
      \item $M$ are other operating expenditure/price index.
  \end{itemize}
  \item observed from 2002-2008
\end{itemize}
}


\section{Order-m or $\alpha$-frontiers ?}
\frame[t] {
  Why using order-m or $\alpha$-frontiers ?
\begin{itemize}[<+->]
\item These methods are robust to influential firms or ``outliers''
  \item Could be used to detect (and remove) influential firms or ``outliers'' (Simar, 2003) and then use ``classical'' methods
  \item[$\rightarrow$] We want to keep those firms in the analysis !
  \item Methods do not suffer of \textit{the curse of dimensionality} (more on that later) % see p78 of Daraio Simar book 2007
  \item order-m scores have an economic (and local) interpretation :
  %\item[]$\rightarrow$ it is (input-oriented case)  the ``\emph{expected minimal value  of input achievable among m firms drawn from the population of firms producing at least this level of output}'' (Cazals, Florens and Simar, 2002).

  \item[]$\rightarrow$ it is (output-oriented case) the  ``\emph{expected maximum value of output among m firms drawn from the population of firms using less than this level of inputs}''
  \item $\alpha$-frontiers have a very appealing statistical definition too..
  \item Could be ``fun'' to use  and to compare with other methods on real data.

\end{itemize}
}

\subsection{More precise definition of order-m or $\alpha$-frontiers ?}
\frame[t] {
\begin{itemize}[<+->]
  \item[]The (unknown) production set $\Psi  $ is defined simply as:
\begin{eqnarray*}
        \Psi            &=& \left\{ \left( x,y\right) |\;x\; can \; produce \; y \right\} \\
  \end{eqnarray*}
  \item[] Can be approximated by several Sets :
\begin{eqnarray*}
\widehat{\Psi}_{FDH}    &=& \left\{ \left( x,y\right) | y \leq Y_i ,\;  x \geq X_i, \; i=1, \cdots n \right\} \\
 \widehat{\Psi}_{DEA}    &=& \left\{ \left( x,y\right) |  \;y\leq \sum_{i =1}^n \gamma_{i} Y_{i},x\geq \sum_{i =1}^n  \gamma_{i} X_{i}, \gamma_{i }\geq 0 ; \sum_{i =1}^n \gamma_{i} =1 \right\} \\
  %\Psi_m (x)            &=& E \left [ max(Y^1, \cdots, Y^m) | X  \leq x   \right ] \\
\end{eqnarray*}
\end{itemize}
}

\subsection{A different way of defining efficiency scores(Daouia-Simar, 2007) }

\frame[t] {
\begin{itemize}[<+->]
  \item[$\Psi$] Consider $H_{X,Y}(x,y)= Prob( X  \leq x , Y \geq y)$ which is the probability for a firm operating at level (x,y) to be dominated.
  \item In short:  $(x,y) \in \Psi \Leftrightarrow H_{X,Y}(x,y) >0 $
  \item[]
  \item[] Note that: $H_{X,Y}(x,y)= Prob( X  \leq x , Y \geq y)$ can be decomposed:
  \begin{eqnarray*}
   = Prob( X  \leq x | Y \geq y) Prob(Y \geq y )  &=& F_c(x|y)S_Y(y) \\
   = Prob(Y \geq y | X  \leq x ) Prob(X  \leq x )  &=& S_c(y|x)F_X(x) \\
   \end{eqnarray*}
   \item For all $x$ such that $ F_X(x)>0 $, the (Farell) output-efficiency score can be defined as (Daraio \& Simar, 2006):
   \begin{eqnarray*}
    \lambda(x,y) &=& sup\{\lambda | H_{X,Y}(x,\lambda y) >0 \}\\
                &=& sup\{\lambda | S_c( \lambda y|x) >0 \} \\
   \end{eqnarray*}
\end{itemize}
}


\frame[t] {
\begin{itemize}[<+->]
   \item[$\alpha$] So if $\lambda(x,y)$  is the (Farell) output-efficiency score :
   \begin{eqnarray*}
    \lambda(x,y) &=& sup\{\lambda | H_{X,Y}(x,y) >0 \} \\
                &=& sup\{\lambda | S_c(x,y) >0 \} \\
   \end{eqnarray*}
   \item For all $x$ such that $ F_X(x)>0 $, the $\alpha$\textit{-quantile  output efficiency score} is simply defined as (Daouia \& Simar, 2007) :
   \begin{eqnarray*}
    \lambda_{\alpha}(x,y)    &=& sup \{\lambda | S_c( \lambda y|x) > 1- \alpha \} \\
   \end{eqnarray*}
   \item and so the $\alpha$\textit{-frontier (output-oriented case)} is the set:
    \begin{eqnarray*}
    \Psi_{\alpha}  = \{ x, \lambda(x,y)\cdot y    \; \; s.t. \;  (x,y) \in \Psi \} \\
   \end{eqnarray*}

\end{itemize}
}



\frame[t] {
\begin{itemize}[<+->]
  \item[order-m] The production set can be described in terms of its sections. $ \forall x$, the frontier in the output direction $\Psi(x)$:
  \begin{eqnarray*}
  \Psi(x)  & =& sup \{ y | (x,y) \in \Psi \} \\
       &= & sup \{y | Sc(y|x) > 0 \} \\
   ( &= & sup \{y | Prob(Y \geq y | X  \leq x )  > 0 \} )
 \end{eqnarray*}
    \item given a fixed integer $m$,  for a given level of input $x$, the  \textit{order-m frontier in the output direction} $\Psi_m(x)$ is:
    \begin{eqnarray*}
         \Psi_m (x)     = E \left[ max(Y^1, \cdots, Y^m) | X  \leq x   \right]
    \end{eqnarray*}
    \item it is the expected maximum value of $m$ random variables $Y^1, \cdots, Y^m$  of $m$ firms drawn from the population of firms using using less than the level $x$ of input.
    \item And so the \textit{order-m output efficiency} measure is
    \begin{eqnarray*}
        \lambda_{m}(x,y) = sup\{\lambda | (x, \lambda y) \in \Psi_m(x) \}
     \end{eqnarray*}
\end{itemize}

}


\section{Airlines industry (in 2-D)}



\frame[t] {
  \vspace{1cm}
Efficiency over time : 3 questions
\begin{itemize}[<+->]
  \item How to measure efficiency each year ? Over years ?
  \item[$\rightarrow$]  Using distance towards a frontier (possibly an order-m or $\alpha$-frontier)
  \item How to visualize/measure changes over time ?
  \item[$\rightarrow$]  Question addressed in this paper
  \item How to disentangle the effects of technical change vs productivity changes ?
  \item[$\rightarrow$]  Using decomposition of productivity index (Simar-Wilson 98, Wheelock Wilson,  1999 or  F\"{a}re, Grosskopft \& Sickles 2007)

\end{itemize}
}


\subsection{Airlines}

\frame[t] {
\begin{itemize}
  \item We choose here to represent the frontiers \textit{as if} $K$ was the only input in 2002 and (with LCC highlighted ).
     \includegraphics[width=0.7\textwidth]{Graphics/GraphR-021}
\end{itemize}
}


\frame[t] {
\begin{itemize}
  \item Computing the efficiency gives a ranking of airlines
  \item10 first airlines efficiency, year 2002 (output oriented, 4 inputs, sorted by $\alpha$-scores).
\end{itemize}
{\footnotesize
\begin{tabular}{rllllll}
  \hline
 & code & Airline & DEA & FDH & M-Score & $\alpha$-Score \\
  \hline
1 & 5X & UNITED PARCEL & 1 & 1 & 1.186 & 2.569 \\
  2 & KL & KLM & 1 & 1 & 1.244 & 2.418 \\
  3 & AS & ALASKA & 1 & 1 & 1.131 & 1.651 \\
  4 & AF & AIR FRANCE & 1 & 1 & 1.177 & 1.577 \\
  5 & TZ & ATA AIRLINES & 0.77 & 1 & 1.088 & 1.516 \\
  6 & US & US AIRWAYS & 1 & 1 & 1.161 & 1.415 \\
  7 & IB & IBERIA & 0.868 & 1 & 1.113 & 1.39 \\
  8 & DL & DELTA & 1 & 1 & 1.129 & 1.356 \\
  9 & CO & CONTINENTAL & 0.951 & 1 & 1.124 & 1.332 \\
  10 & FX & FEDERAL EXPRESS & 1 & 1 & 1.139 & 1.285 \\
   \hline
\end{tabular}
}
}


\frame[t] {
\begin{itemize}
   \item 10 first airlines efficiency, year 2002 (output oriented, K unique input, sorted by $\alpha$-scores).
\end{itemize}
{\footnotesize
\begin{tabular}{rllllll}
  \hline
 & code & Airline & DEA & FDH & M-Score & $\alpha$-Score \\
  \hline
1 & KL & KLM & 0.79 & 1 & 1.192 & 1.739 \\
  2 & IB & IBERIA & 0.549 & 1 & 1.113 & 1.39 \\
  3 & DL & DELTA & 0.808 & 1 & 1.129 & 1.356 \\
  4 & FX & FEDERAL EXPRESS & 1 & 1 & 1.139 & 1.285 \\
  5 & US & US AIRWAYS & 0.87 & 1 & 1.105 & 1.252 \\
  6 & AF & AIR FRANCE & 0.681 & 1 & 1.081 & 1.185 \\
  7 & WN & SOUTHWEST & 0.786 & 1 & 1.096 & 1.13 \\
  8 & HP & AMERICA WEST & 0.613 & 1 & 1.043 & 1.103 \\
  9 & CO & CONTINENTAL & 0.714 & 1 & 1.056 & 1.063 \\
  10 & AS & ALASKA & 0.767 & 1 & 1.046 & 1.061 \\
   \hline
\end{tabular}
}
}


\frame[t] {
\begin{itemize}
  \item We choose here to represent the frontiers \textit{as if} $K$ was the only input in 2005 and (with LCC highlighted ).
     \includegraphics[width=0.7\textwidth]{Graphics/GraphR-023}
\end{itemize}
}


\frame[t] {
\begin{itemize}
  \item Computing the efficiency gives a ranking of airlines
  \item 10 first airlines efficiency, year 2005 (output oriented, 4 inputs, sorted by $\alpha$-scores).
\end{itemize}
{\footnotesize
\begin{tabular}{rllllll}
  \hline
 & code & Airline & DEA & FDH & M-Score & $\alpha$-Score \\
  \hline
1 & KL & KLM & 0.79 & 1 & 1.192 & 1.739 \\
  2 & IB & IBERIA & 0.549 & 1 & 1.113 & 1.39 \\
  3 & DL & DELTA & 0.808 & 1 & 1.129 & 1.356 \\
  4 & FX & FEDERAL EXPRESS & 1 & 1 & 1.139 & 1.285 \\
  5 & US & US AIRWAYS & 0.87 & 1 & 1.105 & 1.252 \\
  6 & AF & AIR FRANCE & 0.681 & 1 & 1.081 & 1.185 \\
  7 & WN & SOUTHWEST & 0.786 & 1 & 1.096 & 1.13 \\
  8 & HP & AMERICA WEST & 0.613 & 1 & 1.043 & 1.103 \\
  9 & CO & CONTINENTAL & 0.714 & 1 & 1.056 & 1.063 \\
  10 & AS & ALASKA & 0.767 & 1 & 1.046 & 1.061 \\
   \hline
\end{tabular}

}
}

\frame[t] {
\begin{itemize}
  \item Computing the efficiency gives a ranking of airlines
  \item10 first airlines efficiency, year 2005 (output oriented, 1 input, sorted by $\alpha$-scores).
\end{itemize}
{\footnotesize
\begin{tabular}{rllllll}
  \hline
 & code & Airline & DEA & FDH & M-Score & $\alpha$-Score \\
  \hline
1 & WN & SOUTHWEST & 0.779 & 1 & 1.342 & 1.919 \\
  2 & NH & ALL NIPPON AIRWAYS & 0.914 & 1 & 1.368 & 1.642 \\
  3 & LX & SWISS & 0.985 & 1 & 1.263 & 1.636 \\
  4 & IB & IBERIA & 0.639 & 1 & 1.263 & 1.576 \\
  5 & OS & AUSTRIAN AIRLINES GROUP & 1 & 1 & 1.2 & 1.573 \\
  6 & BD & BRITISH MIDLAND & 0.864 & 1 & 1.125 & 1.515 \\
  7 & CO & CONTINENTAL & 0.764 & 0.938 & 1.135 & 1.465 \\
  8 & US & US AIRWAYS & 0.722 & 0.951 & 1.131 & 1.347 \\
  9 & AF & AIR FRANCE & 0.948 & 1 & 1.184 & 1.328 \\
  10 & FX & FEDERAL EXPRESS & 1 & 1 & 1.169 & 1.306 \\
   \hline
\end{tabular}

}
}


%\section{Efficiency over time}
% \frame[t] {
%  \vspace{1cm}
%Visualizing data over time is difficult
%\begin{itemize}
%  \item Let's try with R-googleVis
%\end{itemize}
%}

\frame[t] {
\begin{itemize}
\item[] Efficiency scores in 2002
\end{itemize}
{\footnotesize
\begin{tabular}{lrrrrrrrr}
 \textbf{Variable} & $\mathbf{\bar{x}}$ & \textbf{Min} & $\mathbf{\widetilde{x}}$ & \textbf{Max} & $\mathbf{n}$ & $\bar{Nb}_{Eff}$ & $\bar{Nb}_{Super}$ & $\bar{X}_{not}$ \\
  \hline
\multicolumn{2}{l}{With 4 inputs} & \multicolumn{2}{c}{}\\
DEA (vrs) & 0.79 & 0.25 & 0.86 & 1.00 & 56 & 18 &  0 & 0.69 \\
FDH & 0.98 & 0.56 & 1.00 & 1.00 & 56 & 50 &  0 & 0.84 \\
m-scores & 1.03 & 0.57 & 1.02 & 1.24 & 56 & 11 & 39 & 0.85 \\
$\alpha$-scores & 1.12 & 0.56 & 1.00 & 2.57 & 56 & 33 & 17 & 0.84 \\ \hline
\multicolumn{2}{l}{With One input} & \multicolumn{2}{c}{}\\
DEA (vrs) &  0.47 & 0.02 & 0.43 & 1.00 & 56 &  3 &  0 & 0.44 \\
FDH & 0.73 & 0.04 & 0.88 & 1.00 & 56 & 23 &  0 & 0.54 \\
 m-scores & 0.76 & 0.04 & 0.91 & 1.19 & 56 &  4 & 20 & 0.54 \\
$\alpha$-scores & 0.81 & 0.04 & 0.93 & 1.74 & 56 & 15 & 11 & 0.55 \\ \hline\hline
\end{tabular}
}
}



\frame[t] {
\begin{itemize}
\item[] Efficiency scores in 2005
\end{itemize}
{\footnotesize
\begin{tabular}{lrrrrrrrr}
 \textbf{Variable} & $\mathbf{\bar{x}}$ & \textbf{Min} & $\mathbf{\widetilde{x}}$ & \textbf{Max} & $\mathbf{n}$ & $\bar{Nb}_{Eff}$ & $\bar{Nb}_{Super}$ & $\bar{X}_{not}$ \\
  \hline
\multicolumn{2}{l}{With 4 inputs} & \multicolumn{2}{c}{}\\
DEA (vrs) & 0.69 & 0.18 & 0.70 & 1.00 & 103 & 20 &  0 & 0.62 \\
FDH & 0.94 & 0.35 & 1.00 & 1.00 & 103 & 82 &  0 & 0.71 \\
m-scores  & 1.03 & 0.38 & 1.02 & 1.66 & 103 & 14 & 69 & 0.73 \\
$\alpha$-scores & 1.16 & 0.35 & 1.00 & 3.49 & 103 & 53 & 34 & 0.71 \\ \hline
\multicolumn{2}{l}{With One input} & \multicolumn{2}{c}{}\\
DEA (vrs)& 0.36 & 0.01 & 0.30 & 1.00 & 103 &  5 &  0 & 0.33 \\
FDH & 0.50 & 0.01 & 0.46 & 1.00 & 103 & 20 &  0 & 0.38 \\
m-scores & 0.57 & 0.01 & 0.48 & 1.37 & 103 &  6 & 18 & 0.40 \\
$\alpha$-scores & 0.66 & 0.01 & 0.61 & 1.92 & 103 & 12 & 16 & 0.45 \\ \hline\hline
\end{tabular}
}
}

\subsection{"Static" Measure of efficiency using Frontiers}
%(Tulkens and Van den Eeckaut, 1995)
\frame[t] {
Here, we  observe firms inputs and outputs  $(X_{it},Y_{it})$ \textbf{over the period} $t=1,\cdots,T$.
\bigskip
\begin{itemize}
    \item We have several {\color{brique} contemporaneous} production possibilities set:
    \begin{equation*}
        \Psi_{t}^{cont}=\left\{ \left( x,y\right) |\;y\leq \sum_{i \in S(t)}\;\gamma_{it} Y_{it},\;x\geq \sum_{i \in S(t)}\;\gamma_{it} X_{it} ; \gamma_{i\tau }\geq 0\right\} .
    \end{equation*}
    \item $S(\tau)$ is the set of firms operating at time $\tau$
    \end{itemize}
}

\subsection{How to measure technical change over time ?}

\frame[t] {
Questions:
\begin{itemize}
    \item Frontier shifts are interesting (technical progress)  and cannot be ``seen'' if more than one input.
    \item Frontiers could be affected by `` outliers'' $\rightarrow$ Could use a more robust frontier estimator
    \item  Scores at date 1 and date 2 not comparable.
    \item[$\rightarrow$] Efficiency is a relative that changes with production sets (and frontiers).
\end{itemize}
}

\frame[t] {
\begin{itemize}
    \item Efficiency scores of airlines on contemporaneous DEA frontiers (1 input)
    \begin{figure}[t]
       \includegraphics[width=0.6\textwidth]{Graphics/GraphR-034.pdf}
    \end{figure}
  \item
\end{itemize}
}


\frame[t] {
\begin{itemize}
    \item Efficiency scores of airlines on contemporaneous $\alpha$-frontiers (1 input)
    \begin{figure}[t]
       \includegraphics[width=0.6\textwidth]{Graphics/GraphR-037.pdf}
    \end{figure}
 
\end{itemize}
}


\section{All Time Production Set }
\frame[t] {
We propose to compute the all-time empirical production set (ATPS) as the union of all the production sets $t=1, \cdots, T$ :
   \bigskip
\begin{itemize}[<+->]
\item[]
\begin{eqnarray*}
\Psi^{ATPS} =   \bigcup_{t=1}^{T} \Psi_{t}^{cont}
\end{eqnarray*}

\item[] {\footnotesize $S(\tau)$ is the set of firms operating at time $\tau$ }
{\footnotesize
\begin{eqnarray*}
\Psi^{ATPS} =  \bigcup_{t=1}^{T} \left\{ \left( x,y\right) |\;y\leq \sum_{i \in S(t)}\;\gamma_{it} Y_{it},\;x\geq \sum_{i \in S(t)}\;\gamma_{it} X_{it} ; \gamma_{i\tau }\geq 0\right\} .
\end{eqnarray*}
}
\item[] or
{\footnotesize
\begin{eqnarray*}
\Psi^{ATPS} =  \left\{ \left(x,y\right) | \; y \leq {\color{blue} {\sum_{\tau = 1}^{T}}} \sum_{j \in S(\tau)} z_{j\tau} Y_{j\tau} ,
 x \geq {\color{blue}\sum_{\tau = 1}^{T} } \sum_{j \in S(\tau)}  z_{j\tau} X_{j\tau} ,   \mbox{all} \; z_{j\tau} \geq 0 \right\}.
\end{eqnarray*}
}
\end{itemize}
}

\frame[t] {
\begin{center}
{\color{brique} $ \Psi_{1}^{cont} $}
\end{center}

\begin{figure}[t]
\includegraphics[width=0.7\textwidth]{Graphics/PY1.pdf}
\end{figure}
}


\frame[t] {
\begin{center}
{\color{blue} $ \Psi^{ATPS} $}
\end{center}
\begin{figure}[t]
\includegraphics[width=0.7\textwidth]{Graphics/PY1Y2.pdf}
\end{figure}
}



\subsection{How to measure technical change over time ?}

\frame[t] {
Computing firms efficiency score relative to that production set frontier
\begin{itemize}
    \item The  $\Psi^{ATPS}$ production set  does not move over time.
    \item Computing firms efficiency each year gives a visualisation of the efficiency changes over time.
    \item Instead of using the DEA frontier on the $\Psi^{ATPS}$ frontier we use the order-m or $\alpha$ frontier.
\end{itemize}
}


\frame[t] {
\begin{itemize}
    \item Current year efficiency scores on $\Psi^{ATPS}$  DEA frontiers
    \begin{figure}[t]
        \includegraphics[width=0.6\textwidth]{Graphics/GraphR-039.pdf}
    \end{figure}
  \item[] Output oriented DEA-efficiency scores with $K$ as unique input.
\end{itemize}
}

\frame[t] {
\begin{itemize}
    \item Current year efficiency scores  on $\Psi^{ATPS}$  $\alpha$-frontier
    \begin{figure}[t]
        \includegraphics[width=0.6\textwidth]{Graphics/GraphR-042.pdf}
    \end{figure}
  \item[] Output oriented $\alpha$-efficiency scores with $K$ as unique input.
\end{itemize}
}




\frame[t] {
\begin{itemize}
    \item Current year efficiency scores on $\Psi^{ATPS}$  DEA frontiers
    \begin{figure}[t]
        \includegraphics[width=0.6\textwidth]{Graphics/GraphR-039.pdf}
    \end{figure}
  \item[] Output oriented DEA-efficiency scores with  4 inputs.
\end{itemize}
}

\frame[t] {
\begin{itemize}
    \item Current year efficiency scores  on $\Psi^{ATPS}$  $\alpha$-frontier
    \begin{figure}[t]
        \includegraphics[width=0.6\textwidth]{Graphics/GraphR-042.pdf}
    \end{figure}
  \item[] Output oriented $\alpha$-efficiency scores with  4 inputs.
\end{itemize}
}




\subsection{Results ?}
\frame[t] {
Computing firms efficiency score relative to that frontier reveals:
\begin{itemize}
    \item The efficiency of firms increase year after year
    \item Computing order-m or $\alpha$ frontier gives the same feature, but with some changes in the details (see year 2008)
    \item Some evidence on technical progress ( \textit{i.e.} shift upwards of the frontier)
    \item Rsults not clear with 4 inputs.
    \end{itemize}
}


% Sur le graphique, on a l'impression que la productivit� baisse en 2008 (version DEA) et pas sur order-m (� voir)
\section{Malmquist decomposition}

\subsection{Simar \& Wilson (1998), Wheelock \& Wilson, 1999)}
\frame[t] {
For each subperiod $[t_1, t_2]$, we compute the Malmquist Index (MI) on a balanced panel using
 $(X_i, Y_i)_{t=t_1;t_2}$. We decompose MI into different elements.
{\footnotesize
\begin{eqnarray*}
MI &=& \mbox{Pure efficiency change} \times \mbox{Change in the scale
efficiency} \\
& \times & \mbox{Pure change in technology} \\
& \times & \mbox{Change in the scale of the technology}
\end{eqnarray*}

\begin{eqnarray}
MI&=& \left(\frac{D^{VRS}_\mathbf{t_2}(x_\mathbf{t_2}, y_\mathbf{t_2})}{D^{VRS}_%
\mathbf{t_1}(x_\mathbf{t_1}, y_\mathbf{t_1})}\right) \times \left(\frac{{D^{CRS}_%
\mathbf{t_2}(x_\mathbf{t_2}, y_\mathbf{t_2})} \;/\; {D^{VRS}_\mathbf{t_2}(x_\mathbf{c%
}, y_\mathbf{t_2})}}{{D^{CRS}_\mathbf{t_1}(x_\mathbf{t_1}, y_\mathbf{t_1})} \;/\; {%
D^{VRS}_\mathbf{t_1}(x_\mathbf{t_1}, y_\mathbf{t_1})}}\right)  \notag \\
& \times& \left(\frac{D^{VRS}_\mathbf{t_1}(x_\mathbf{t_2}, y_\mathbf{t_2})}{%
D^{VRS}_\mathbf{t_2}(x_\mathbf{t_2}, y_\mathbf{t_2})} \times \frac{D^{VRS}_\mathbf{%
t_1}(x_\mathbf{t_1}, y_\mathbf{t_1})}{D^{VRS}_\mathbf{t_2}(x_\mathbf{t_1}, y_\mathbf{t_1}%
)} \right)^{0.5}  \notag \\
& \times& \left( \frac{D^{CRS}_\mathbf{t_1}(x_\mathbf{t_2}, y_\mathbf{t_2}) \;/\;
D^{VRS}_\mathbf{t_1}(x_\mathbf{t_2}, y_\mathbf{t_2})} {D^{CRS}_\mathbf{t_2}(x_%
\mathbf{t_2}, y_\mathbf{t_2})\;/\; D^{VRS}_\mathbf{t_2}(x_\mathbf{t_2}, y_\mathbf{t_2})%
} \times \frac{D^{CRS}_\mathbf{t_1}(x_\mathbf{t_1}, y_\mathbf{t_1})\;/\;D^{VRS}_%
\mathbf{t_1}(x_\mathbf{t_1}, y_\mathbf{t_1})} {D^{CRS}_\mathbf{t_2}(x_\mathbf{t_1}, y_%
\mathbf{t_1}) \;/\; D^{VRS}_\mathbf{t_2}(x_\mathbf{t_1}, y_\mathbf{t_1})}
\right)^{0.5}
\end{eqnarray}}
}


\frame[t] {
Malmquist decomposition on our sample for several subperiods (Simar Wilson 1998,  Wheelock Wilson 1999)\\

 \textbf{Change the subperiods, here 1 input ??}\\

{\footnotesize
\begin{tabular}{rrrrrrrr}
  \hline
    year1 & year2 & Malm & Pure Eff & Scale & Pure Tech & ScaleTech  \\
  \hline
   2002 & 2003 & 1.13 & 0.99 & 0.99 & 1.17 & 1.00 \\
   2003 & 2004 & 1.27 & 0.96 & 0.90 & 1.31 & 1.12 \\
   2004 & 2005 & 1.04 & 0.86 & 1.06 & 1.23 & 0.97  \\
   2005 & 2006 & 1.11 & 1.23 & 1.08 & 0.96 & 0.94  \\
   2006 & 2007 & 1.13 & 1.08 & 0.97 & 1.11 & 1.01 \\
   2007 & 2008 & 1.30 & 0.96 & 0.98 & 1.06 & 1.01 \\ \hline
   2002 & 2008 & 1.85 & 1.03 & 0.95 & 1.96 & 1.01 \\
   \hline
\end{tabular}
}

}

\subsection{Conclusion \& todo list}

\frame[t] {
Conclusion : \\
Lot of work remains (also on the data!)
\begin{itemize}[<+->]
  \item Order-m and $\alpha$-frontier difficult to interpret with sparse data and/or more than one input
  \item[] $\rightarrow$ need of an input index.
  \item Nice properties and easy to use (in 2-D)
  \item Many questions for airlines industry :
  \item[-] Seems to be evidence of technical progress  ?
  \item[] $\rightarrow$ decomposition of productivity index could be done differently (see O'Donnell 2012).
  \item Use conditional version of efficiency measure $\lambda(x, y|z)$ where $z$ could be a measure of airlines degree of competition.
\end{itemize}
}


\end{document}



